\documentclass{article}
\usepackage{amsmath}
\usepackage{amssymb}
\usepackage{graphicx}
\usepackage{color}
\usepackage{enumitem}
\usepackage[margin=0.5in]{geometry}
\usepackage{tabto}
\usepackage{dsfont}


\title{MATH 318, Assignment 4}
\author{Kyle Rubenok, 260667187}
\date{October 2019}

\begin{document}

\maketitle

    \begin{enumerate}
        \item 
            \begin{enumerate}[label=\arabic*)] 
                \item The atoms of B are as follows: 
                    \newline 
                    $p \wedge q \wedge r$ \tab $p \wedge q \wedge \neg r$ \newline
                    $\neg p \wedge \neg q \wedge r$ \tab $\neg p \wedge \neg q \wedge \neg r$ \newline
                    $p \wedge \neg q \wedge r$ \tab $p \wedge \neg q \wedge \neg r$ \newline
                    $\neg p \wedge q \wedge r$ \tab $\neg p \wedge q \wedge \neg r$ \newline
                \item From the 8 atoms above, we have that there are $2^8$ elements in B. 
            \end{enumerate}
        \item
            \begin{enumerate}[label=\arabic*)]
                \item Take $B = \mathcal{P}$. Each of the atoms in B makeup the singleton set with every non-empty set in B containing a singleton. This means that for every nonzero $b \in B$, there is an atom $a \in B$ with $ a \leq b$. 
                \item Yes, take B as any boolean algebra without atoms. Additionally let B' be the algebra 2. From this, $B \times B'$ has exactly one atom, $(0,1)$.
            \end{enumerate}
        \item Question 3
        \item  
            \begin{enumerate}[label=\arabic*)]
                \item To show that $|$ is a partial order on $\mathds{N}$ we must show that it is transitive. We know that it is reflexive since $n=1 \cdot n$ so $n|n$. Next, suppose that $n|m$ and $m|n$. For some $i$ and $j$, we can then see that $m=in$ and $n=jm$ giving that $ m = in = ijm$. If m = 0, $n=j \cdot 0 = 0$ and $n=m$. Alternatively, $ij = 1$ and $i = j = 1$ giving that $n = 1 \cdot m = m$ showing it is antisymmetric. Now, if $n|m$ and $m | k$ then $m = in$ and $k = jm$ for some $i$ $j$ therefore $k = jm = ijn$. Giving that $n|p$ is transitive and a partial order. 
                \item We have that $ n = n \cdot 1$ for any n, therefore $1|n$ and 1 a least element.
                \item We have that $ 0 = 0 \cdot n$ for any n, therefore $n|0$ and 0 a greatest element.
            \end{enumerate}
        \item We need two chains to cover P since P is not itself a chain ($2\nmid 3$ and $3\nmid 2$). Allow $C_1 = \{1,3\}$ and $C_2 = \{2,4\}$. Both $C_1$ and $C_2$ are chains and $P=C_1\cup C_2$. This means the smallest number of chains that covers P is 2. 
        \item One potential solution
        $$\o \textless \{1\}\textless \{2\}\textless \{3\}\textless \{1,2\}\textless \{1,3\}\textless \{2,3\}\textless \{1,2,3\} $$
        \item Take $B \in mathcal{P}(\mathds{N})$ with the property that $B \subset \{1,2\}$ and $B \subset \{1,3\}$. This gives that $B \subset (\{1,2\} \cap \{1,3\})$ This means that $B \subset \{1\}$ and that $B \notin P$. Since $\{1,2\}$ and $\{1,3\}$ are in $P$ without a lower bound in $P$, $P$ also does not have a lower bound. 
        \item 
    \end{enumerate}
\end{document}
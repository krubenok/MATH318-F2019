\documentclass{article}
\usepackage{amsmath}
\usepackage{amssymb}
\usepackage{graphicx}
\usepackage{color}
\usepackage{enumitem}
\usepackage[margin=0.5in]{geometry}
\usepackage{tabto}


\title{MATH 318, Assignment 4}
\author{Kyle Rubenok, 260667187}
\date{October 2019}

\begin{document}

\maketitle

    \begin{enumerate}
        \item 
            \begin{enumerate}[label=\arabic*)] 
                \item The atoms of B are as follows: 
                    \newline 
                    $p \wedge q \wedge r$ \tab $p \wedge q \wedge \neg r$ \newline
                    $\neg p \wedge \neg q \wedge r$ \tab $\neg p \wedge \neg q \wedge \neg r$ \newline
                    $p \wedge \neg q \wedge r$ \tab $p \wedge \neg q \wedge \neg r$ \newline
                    $\neg p \wedge q \wedge r$ \tab $\neg p \wedge q \wedge \neg r$ \newline
                \item From the 8 atoms above, we have that there are $2^8$ elements in B. 
            \end{enumerate}
        \item
            \begin{enumerate}[label=\arabic*)]
                \item Take $B = \mathcal{P}$. Each of the atoms in B makeup the singleton set with every non-empty set in B containing a singleton. This means that for every nonzero $b \in B$, there is an atom $a \in B$ with $ a \leq b$. 
                \item Yes, take B as any boolean algebra without atoms. Additionally let B' be the algebra 2. From this, $B \times B'$ has exactly one atom, $(0,1)$.
            \end{enumerate}
        \item 
    \end{enumerate}
\end{document}